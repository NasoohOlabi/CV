%%%%%%%%%%%%%%%%%
% This is an example CV created using altacv.cls (v1.6.4, 13 Nov 2021) written by
% LianTze Lim (liantze@gmail.com), based on the
% Cv created by BusinessInsider at http://www.businessinsider.my/a-sample-resume-for-marissa-mayer-2016-7/?r=US&IR=T
%
%% It may be distributed and/or modified under the
%% conditions of the LaTeX Project Public License, either version 1.3
%% of this license or (at your option) any later version.
%% The latest version of this license is in
%%    http://www.latex-project.org/lppl.txt
%% and version 1.3 or later is part of all distributions of LaTeX
%% version 2003/12/01 or later.
%%%%%%%%%%%%%%%%

%% Use the "normalphoto" option if you want a normal photo instead of cropped to a circle
% \documentclass[10pt,a4paper,normalphoto]{altacv}

\documentclass[10pt,a4paper,ragged2e,withhyper]{altacv}

%% AltaCV uses the fontawesome5 package.
%% See http://texdoc.net/pkg/fontawesome5 for full list of symbols.

% Change the page layout if you need to
\geometry{left=1.25cm,right=1.25cm,top=1.5cm,bottom=1.5cm,columnsep=1.2cm}

% The paracol package lets you typeset columns of text in parallel
\usepackage{paracol}


% Change the font if you want to, depending on whether
% you're using pdflatex or xelatex/lualatex
\ifxetexorluatex
  % If using xelatex or lualatex:
  \setmainfont{Lato}
\else
  % If using pdflatex:
  \usepackage[default]{lato}
\fi

% Change the colours if you want to
\definecolor{VividPurple}{HTML}{313f7e}
\definecolor{SlateGrey}{HTML}{2E2E2E}
\definecolor{LightGrey}{HTML}{666666}
\definecolor{Grey}{HTML}{ebf2f2}
% \colorlet{name}{black}
\colorlet{tagline}{VividPurple}
\colorlet{heading}{VividPurple}
\colorlet{headingrule}{VividPurple}
% \colorlet{subheading}{PastelRed}
\colorlet{accent}{VividPurple}
\colorlet{emphasis}{SlateGrey}
\colorlet{body}{LightGrey}
\pagecolor{Grey}

% Change some fonts, if necessary
% \renewcommand{\namefont}{\Huge\rmfamily\bfseries}
% \renewcommand{\personalinfofont}{\footnotesize}
% \renewcommand{\cvsectionfont}{\LARGE\rmfamily\bfseries}
% \renewcommand{\cvsubsectionfont}{\large\bfseries}

% Change the bullets for itemize and rating marker
% for \cvskill if you want to
\renewcommand{\itemmarker}{{\small\textbullet}}
\renewcommand{\ratingmarker}{\faCircle}

%% Use (and optionally edit if necessary) this .tex if you
%% want to use an author-year reference style like APA(6)
%% for your publication list
% \input{pubs-authoryear}

%% Use (and optionally edit if necessary) this .tex if you
%% want an originally numerical reference style like IEEE
%% for your publication list
% \input{pubs-num}


\begin{document}
\name{Nasouh AlOlabi}
\tagline{Junior Full Stack Developer}

\photoR{2.5cm}{profile}
% \photoL{2cm}{Yacht_High,Suitcase_High}
\personalinfo{%
   % Not all of these are required!
   % You can add your own with \printinfo{symbol}{detail}
   \email{nasooholabi@gmail.com}
   \phone{09448 00 963}
   % \mailaddress{Damascus, Dumar, sec 16}
   \location{Dumar, Damascus}
   % \homepage{marissamayr.tumblr.com}
   % \twitter{@marissamayer}
   \linkedin{linkedin.com/in/nasooh-olabi/}
   \github{github.com/NasoohOlabi/} % I'm just making this up though.
   %   \orcid{0000-0000-0000-0000} % Obviously making this up too.
   %% You can add your own arbitrary detail with
   %% \printinfo{symbol}{detail}[optional hyperlink prefix]
   % \printinfo{\faPaw}{Hey ho!}
   %% Or you can declare your own field with
   %% \NewInfoFiled{fieldname}{symbol}[optional hyperlink prefix] and use it:
   % \NewInfoField{gitlab}{\faGitlab}[https://gitlab.com/]
   % \gitlab{your_id}
   %%
   %% For services and platforms like Mastodon where there isn't a
   %% straightforward relation between the user ID/nickname and the hyperlink,
   %% you can use \printinfo directly e.g.
   % \printinfo{\faMastodon}{@username@instace}[https://instance.url/@username]
   %% But if you absolutely want to create new dedicated info fields for
   %% such platforms, then use \NewInfoField* with a star:
   % \NewInfoField*{mastodon}{\faMastodon}
   %% then you can use \mastodon, with TWO arguments where the 2nd argument is
   %% the full hyperlink.
   % \mastodon{@username@instance}{https://instance.url/@username}
}

\makecvheader

%% Depending on your tastes, you may want to make fonts of itemize environments slightly smaller
\AtBeginEnvironment{itemize}{\small}

%% Set the left/right column width ratio to 6:4.
\columnratio{0.6}

% Start a 2-column paracol. Both the left and right columns will automatically
% break across pages if things get too long.
\begin{paracol}{2}

   \cvsection{Courses \& Training}

   \cvevent{Odoo Framework \& ERP}{EBTECH}{Febuary 2022 -- April 2022}{}
   \begin{itemize}
      \item Use Odoo Sales, Accounting, Inventory, CRM modules.
      \item Create and maintain an Odoo ERP module.
      \item Take full advantage of the Odoo ORM Layer and its features.
      \item Build User Friendly UI using Odoo's built-in xml web views.
   \end{itemize}

   \divider

   \cvevent{Data Structures and Algorithms Specialization}{UCSanDiego}{}{}
   \begin{itemize}
      \item Wrote and debugged low level C++ advanced assignments.
      \item Worked on a high velocity Capstone Project that performed genome assembling.
      \item Briefly Touched on Streams and query point algorithms.
      \item Reenforced my what I Learned in Accelerated Computer Science Fundamentals By The University of Illinois at Urbana-Champaign.
   \end{itemize}

   \divider

   \cvevent{Google IT Automation with Python}{Google}{}{}
   \begin{itemize}
      \item Learned about Git (VCS) Interacting with the Operating System, Debugging and Configuration Management.
      \item Worked on Automating Real-World Task Capstone using Python Batteries Included Ideology.
      \item Reenforced my what I Learned in Python 3 Programming Specialization By The University of Michigan where I built an OCR newspaper analyzer as a capstone project.
   \end{itemize}
   \divider

   \cvevent{Developing Applications with Google Cloud}{Google}{}{}

   \begin{itemize}
      \item Understood the fundamental services and concepts of cloud computing.
      \item Learned about various deployment strategies.
   \end{itemize}

   \divider

   \cvevent{Functional Programming
      in Scala}{École polytechnique fédérale de Lausanne (EPFL)}{}{}

   \begin{itemize}
      \item Gained a deep understanding of Scala and how it leverages the best of both worlds the Functional and the Object Oriented.
      \item Enforced my java skills since scala is interoperable with java.
      \item Learned about Reactivity, Parallelism and Spark.
      \item Got comfortable with immutability, Signals, Futures and Promises.
   \end{itemize}


   % \cvevent{Product Engineer}{Google}{23 June 1999 -- 2001}{Palo Alto, CA}

   % \begin{itemize}
   % \item Joined the company as employe \#20 and female employee \#1
   % \item Developed targeted advertisement in order to use user's search queries and show them related ads
   % \end{itemize}

   % \cvsection{A Day of My Life}

   % Adapted from @Jake's answer from http://tex.stackexchange.com/a/82729/226
   % \wheelchart{outer radius}{inner radius}{
   % comma-separated list of value/text width/color/detail}
   % Some ad-hoc tweaking to adjust the labels so that they don't overlap
   % \hspace*{-1em}  %% quick hack to move the wheelchart a bit left
   % \wheelchart{1.5cm}{0.5cm}{%
   %    10/13em/accent!30/Sleeping \& dreaming about work,
   %    25/9em/accent!60/Public resolving issues with Yahoo!\ investors,
   %    5/11em/accent!10/\footnotesize\\[1ex]New York \& San Francisco Ballet Jawbone board member,
   %    20/11em/accent!40/Spending time with family,
   %    5/8em/accent!20/\footnotesize Business development for Yahoo!\ after the Verizon acquisition,
   %    30/9em/accent/Showing Yahoo!\ \mbox{employees} that their work has meaning,
   %    5/8em/accent!20/Baking cupcakes
   % }

   \switchcolumn

   \cvsection{Life Philosophy}
   \begin{quote}
      ``The More You Know, the Easier It Is to Learn.''
   \end{quote}

   \cvsection{Education}

   \cvevent{B.S.\ in Software Engineering}{Higher Institute of Applied Sciences and Technology (HIAST)}{October 2018 -- On going...}{}

   \cvsection{Proud of}

   \cvachievement{\faProjectDiagram}{meScheduler}{Personal mini project on github which helped me learn at first hand about tackling a real world problem using Search Algorithms in a Typescript-React PWA utilizing Webworkers to help out with the complexity of the problem}

   % \divider

   % \cvachievement{\faHandPaper}{State Management with Immutability}{I showed despite the hard moments and my willingness to stay with Yahoo after the acquisition}

   \divider

   \cvachievement{\faTrophy}{Best Programming and Algorithms}{OpenSumo ARC3-Syria 2017}

   % \divider

   % \cvachievement{\faMale}{Inspiring women in tech}{Youngest CEO on Fortune's list of 50 most powerful women}

   \cvsection{Strengths}

   \cvtag{Hard-working (16/24)}
   \cvtag{Persuasive}\\
   \cvtag{Patience}
   \cvtag{Optimism}

   \divider\smallskip

   \cvtag{Javascript}
   \cvtag{React}
   \cvtag{Vscode}
   \cvtag{Scala}
   \cvtag{Java}
   \cvtag{Web Apps \& PWAs}

   \cvsection{Languages}

   \cvskill{English}{5}
   % \divider

   \cvskill{Arabic}{5}
   % \divider

   % \cvskill{German}{3.5} %% supports X.5 values.

   \cvsection{Notes}
   \begin{itemize}
      \item Links to certificates and other details are in my Linkedin account and my Github account.
      \item Mentioned certificates were provided by Coursera.
      \item I'm using \LaTeX{} altacv for this.
   \end{itemize}

   % \divider


   % \divider

   % \cvevent{B.S.\ in Symbolic Systems}{Stanford University}{Sept 1993 -- June 1997}{}

   \newpage

\end{paracol}

\end{document}